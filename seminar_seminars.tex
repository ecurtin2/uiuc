\documentclass{revtex4}
\usepackage{graphicx}
\usepackage{amsmath}
\usepackage{amssymb}
\usepackage{mathtools}
\usepackage{textcomp}
\usepackage{algorithm}

\begin{document}
\title{Seminar on Seminars}
\author{Evan Curtin}
\date{\today}
\maketitle

The seminar on seminars had a few useful tips for giving a good presentation. It covered generic topics like using sans serif fonts and high
contrast colors for text. White backgrounds were encouraged, as was keeping the number of sections small. 
I didn't know rapid changes in light intensity could be painful for some folks. I also didn't think about the bottom portion of the 
slides being cut off. 
The content on slides should be light on text, and the aim should be to convince the audience of something, and not a detailed explanataion. 
focus should be spent on making the talk simpler. 

Avoid complicated data, such as 3D plots and large tables. Preferably avoid tables altogether. Since humans are much better at interpreting images than numbers, most tabular data is better presented as a picture.  Make sure to keep the audience in mind! Apparently Americans spell it acknowledgments. 


Also Don't get cute. 

\end{document}
